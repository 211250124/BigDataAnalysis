\documentclass{article}

\usepackage{amsmath}
\usepackage{graphicx}
\usepackage{lipsum}
\usepackage{ctex}
\usepackage{titlesec}
\usepackage{fontspec}

\setmainfont{SimSun}

\title{新闻分类任务报告}
\author{程智镝}
\date{\today}

\begin{document}
\maketitle

\section*{作业任务}

每条数据包含属性: \texttt{category, headline, authors, link, short\_description, date},一共有42个分类。

\begin{itemize}
    \item[a.] 数据集描述
    \item[b.] 数据处理流程
    \item[c.] 分类模型选择,设计,训练,验证,测试
    \item[d.] 限python或java实现模型
\end{itemize}

\section{数据集描述}
使用的数据集包含了新闻文章的各种属性,共有42个不同的分类。数据集中的内容包括headline(标题)、authors(作者)、link(链接)、short\_description(简介)、date(日期)等。经过数据处理,得到了训练集和测试集,并进行了统计分析。

\section{数据处理流程}
在数据处理阶段,对数据进行了清洗,包括处理缺失值等。然后将数据集分割为训练集和测试集,以便进行模型训练和测试。

\section{分类模型选择及设计}
在分类模型选择方面,选择了逻辑回归模型。使用了TF-IDF进行文本特征提取,并将提取出的特征作为逻辑回归模型的输入。使用Python中的Scikit-learn库来实现模型的训练、验证和测试。

\section{模型训练和评估}
在模型训练阶段,使用训练集对逻辑回归模型进行了训练。之后,对模型进行了评估,包括打印了混淆矩阵和分类报告。混淆矩阵展示了模型的预测结果和真实标签的对比,而分类报告包括了精确度、召回率和F1得分等指标,以及每个类别的支持数。
\subsection{性能指标}

模型在测试集上的性能指标如下:\\
\begin{table}[h]
    \centering
    \begin{tabular}{|l|l|l|l|l|}
    \hline
    Metric & Precision & Recall & F1-Score & Support \\
    \hline
    Accuracy & 0.55 & - & - & 41782 \\
    Macro Avg & 0.51 & 0.37 & 0.41 & 41782 \\
    Weighted Avg & 0.54 & 0.55 & 0.52 & 41782 \\
    \hline
    \end{tabular}
    \caption{模型性能评估}
    \end{table}
    
    tips:"Accuracy" 表示准确率,"Macro Avg" 表示宏平均值,"Weighted Avg" 表示加权平均值。

\begin{table}[h]
    \centering
    \begin{tabular}{|l|l|l|l|l|}
    \hline
    Category & Precision & Recall & F1-Score & Support \\
    \hline
    ARTS & 0.34 & 0.13 & 0.19 & 324 \\
    ARTS \& CULTURE & 0.34 & 0.09 & 0.15 & 288 \\
    BLACK VOICES & 0.45 & 0.31 & 0.36 & 867 \\
    BUSINESS & 0.47 & 0.42 & 0.44 & 1175 \\
    COLLEGE & 0.54 & 0.27 & 0.36 & 234 \\
    COMEDY & 0.56 & 0.39 & 0.46 & 1085 \\
    CRIME & 0.50 & 0.52 & 0.51 & 695 \\
    CULTURE \& ARTS & 0.74 & 0.24 & 0.37 & 218 \\
    DIVORCE & 0.78 & 0.57 & 0.66 & 710 \\
    EDUCATION & 0.41 & 0.21 & 0.28 & 191 \\
    ENTERTAINMENT & 0.50 & 0.72 & 0.59 & 3417 \\
    ENVIRONMENT & 0.53 & 0.16 & 0.25 & 270 \\
    FIFTY & 0.49 & 0.12 & 0.19 & 296 \\
    FOOD \& DRINK & 0.59 & 0.62 & 0.61 & 1313 \\
    GOOD NEWS & 0.40 & 0.13 & 0.20 & 305 \\
    GREEN & 0.35 & 0.26 & 0.30 & 504 \\
    HEALTHY LIVING & 0.31 & 0.19 & 0.24 & 1295 \\
    HOME \& LIVING & 0.69 & 0.64 & 0.66 & 891 \\
    IMPACT & 0.38 & 0.20 & 0.27 & 674 \\
    LATINO VOICES & 0.58 & 0.18 & 0.28 & 209 \\
    MEDIA & 0.55 & 0.34 & 0.42 & 568 \\
    MONEY & 0.49 & 0.26 & 0.34 & 368 \\
    PARENTING & 0.49 & 0.56 & 0.52 & 1785 \\
    PARENTS & 0.40 & 0.23 & 0.29 & 715 \\
    POLITICS & 0.63 & 0.84 & 0.72 & 7045 \\
    QUEER VOICES & 0.73 & 0.60 & 0.66 & 1259 \\
    RELIGION & 0.57 & 0.40 & 0.47 & 502 \\
    SCIENCE & 0.61 & 0.34 & 0.44 & 459 \\
    SPORTS & 0.63 & 0.56 & 0.59 & 1018 \\
    STYLE & 0.47 & 0.16 & 0.23 & 462 \\
    STYLE \& BEAUTY & 0.66 & 0.74 & 0.70 & 1894 \\
    TASTE & 0.38 & 0.12 & 0.18 & 421 \\
    TECH & 0.58 & 0.35 & 0.44 & 458 \\
    THE WORLDPOST & 0.48 & 0.36 & 0.41 & 765 \\
    TRAVEL & 0.60 & 0.71 & 0.65 & 1970 \\
    U.S. NEWS & 0.31 & 0.05 & 0.09 & 264 \\
    WEDDINGS & 0.78 & 0.69 & 0.74 & 766 \\
    WEIRD NEWS & 0.37 & 0.21 & 0.27 & 562 \\
    WELLNESS & 0.45 & 0.71 & 0.55 & 3677 \\
    WOMEN & 0.41 & 0.30 & 0.35 & 694 \\
    WORLD NEWS & 0.44 & 0.29 & 0.35 & 673 \\
    WORLDPOST & 0.46 & 0.20 & 0.28 & 496 \\
    \hline
    \end{tabular}
    \caption{分类报告}
\end{table}



混淆矩阵如下所示:

\[
\begin{bmatrix}
    43 & 15 & 7 & \ldots & 0 & 1 \\
    15 & 27 & 6 & \ldots & 14 & 1 \\
    \ldots & \ldots & \ldots & \ldots & \ldots & \ldots \\
    0 & 1 & 3 & \ldots & 5 & 198 \\
    0 & 0 & 4 & \ldots & 6 & 37 \\
\end{bmatrix}
\]
\texttt{svm,decisionTree:}
另外还实现了svm,决策树的模型分析,结果可运行查看
\end{document}
